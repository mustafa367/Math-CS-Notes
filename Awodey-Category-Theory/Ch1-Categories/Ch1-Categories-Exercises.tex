\documentclass{article} % This begins the document


\usepackage[T1]{fontenc} % Fixes some symbols outside of math mode
\usepackage{amsmath, amssymb, amsthm} % Adds some mathematical symbols
\usepackage{enumerate} % Allows parts to be labelled by letter
\usepackage{graphicx} % Allows including images with \includegraphics
\usepackage{fullpage} % Adjusts boundaries
\usepackage{xcolor} % Allows \textcolor to change colors
\usepackage{bookmark} % Creates an outline
\usepackage{listings}
\usepackage[Export]{adjustbox} % Allows images to align with enumeration on top with valign
% Command for creating new problems
% Each problem appears on its own page
% You can use HomeworkSplitter.py to split the pdf for each problem
\newcommand{\problem}[1]{\newpage\pdfbookmark[section]{#1}{#1}\section*{#1}} 


% Some commands to make certain things easier
\renewcommand{\emptyset}{\varnothing}
\newcommand{\abs}[1]{\left|#1\right|}
\newcommand{\set}[1]{\left\{#1\right\}}
\newcommand{\floor}[1]{\left\lfloor#1\right\rfloor}
\newcommand{\ceil}[1]{\left\lceil#1\right\rceil}
\newcommand{\pair}[1]{\langle#1\rangle}


\title{\textbf{1}\\Categories}
\begin{document}
\maketitle

\section*{Definitions}
\begin{enumerate}
    \item A \textit{category} consists of objects and arrows.
    \\For each arrow $f$, there exists a domain $\text{dom}(f)$ and codomain $\text{cod}(f)$. 
    \\If $A=\text{dom}(f)$ and $B=\text{cod}(f)$, we may write $f:A\rightarrow B$. 
    \\If $f:A\rightarrow B$ and $g:B\rightarrow C$, then there is a composite arrow $g\circ f:A\rightarrow C$.
    \\For each object $A$, there is an identity arrow $1_A:A\rightarrow A$.
    \\Objects and arrows must satisfy the following two equations, assuming each is well-defined:
    \begin{enumerate}
        \item $h\circ (g\circ f)=(h\circ g)\circ f$ (Associativity)
        \item $f\circ 1_A=f=1_B\circ f$ (Unit)
    \end{enumerate}
    \item Given categories \textbf{C} and \textbf{D}, a \textit{functor} $F:\textbf{C}\rightarrow\textbf{D}$ is a mapping of objects to objects and arrows to arrows such that:
    \begin{enumerate}
        \item $F(f:A\rightarrow B)=F(f):F(A)\rightarrow F(B)$
        \item $F(1_A)=1_{F(A)}$
        \item $F(g\circ f)=F(g)\circ F(f)$
    \end{enumerate}
\end{enumerate}

\section*{Exercises}
\begin{enumerate}
	\item \begin{enumerate}
        \item \textbf{Rel} has sets as objects and subsets of $A\times B$ as arrows $A\rightarrow B$. Such subsets are also called relations.
        \\The identity arrow $1_A$ is the relation $\set{\pair{a,a}\in A\times A:a\in A}$.
        \\The composite arrow of $R:A\rightarrow B$ and $S:B\rightarrow C$ is $\set{\pair{a,c}\in A\times C:\exists b(\pair{a,b}\in R\:\&\:\pair{b,c}\in S)}$.
        \\
        \\First we note the category is closed under composition since the composite arrow is a relation from $A$ to $C$.
        \\
        \\Second we verify that Unit holds.
        \begin{align*}
            1_B \circ R &= \set{ \pair{a,b_2} \in A \times B : \exists b_1 (\pair{a,b_1} \in R \:\&\: \pair{b_1,b_2} \in 1_B)}
            \\&=\set{ \pair{a,b_2} \in A \times B : \exists b_1 (\pair{a,b_1} \in R \:\&\: b_1=b_2)}
            \\&=\set{ \pair{a,b} \in A \times B : \exists b (\pair{a,b} \in R)}
            \\&=R
            \\S \circ 1_B &= \set{ \pair{b_2,c} \in B \times C : \exists b_1 (\pair{b_2,b_1} \in 1_B \:\&\: \pair{b_1,c} \in S)}
            \\&=\set{ \pair{b_2,c} \in B \times C : \exists b_1 (b_2=b_1 \:\&\: \pair{b_1,c} \in S)}
            \\&=\set{ \pair{b,c} \in B \times C : \exists b (\pair{b,c} \in S)}
            \\&=S
        \end{align*}
        \\
        \\Finally we verify that Associativity holds. Let $T:C\rightarrow D$.
        \begin{align*}
            \pair{a,d}\in T\circ (S\circ R) &\iff \exists c(\pair{a.c}\in S\circ R \:\&\: \pair{c,d}\in T)
            \\&\iff \exists c(\exists b(\pair{a.b}\in R \:\&\: \pair{b,c}\in S) \:\&\: \pair{c,d}\in T)
            \\&\iff \exists b(\pair{a.b}\in R \:\&\: \exists c(\pair{b,c}\in S \:\&\: \pair{c,d}\in T))
            \\&\iff \exists b(\pair{a.b}\in R \:\&\: \pair{b,d}\in T\circ S)
            \\&\iff \pair{a,d}\in (T\circ S)\circ R
            %(T\circ S)\circ R&=\set{\pair{a,d}\in A\times D:\exists b,c(\pair{a,b}\in R\:\&\: \pair{b,c}\in S\:\&\: \pair{c,d}\in T)}
        \end{align*}
        \item $G:\textbf{Sets}\rightarrow\textbf{Rel}$ is a mapping defined by taking objects to themselves and arrows in \textbf{Sets} as functions to its graph in \textbf{Rel}, that is for a function $f: A\times B$ in $\textbf{Sets}$
        \\$G(f)=\set{\pair{a,f(a)}\in A\times B:a\in A}$
        \\
        \\It immediate that $G$ takes objects in $\textbf{Sets}$ to objects in $\textbf{Rel}$.
        \\Let $f: A \rightarrow B$ be an arrow of \textbf{Sets}. $G(f)$ is by definition a subset of the product of $A \times B$, and hence is an arrow $A \rightarrow B$ in \textbf{Rel}.
        \\
        \\First, we show $G$ preserves domains and codomains.
        \begin{align*}
            G(f: A\rightarrow B)&=\set{\pair{a,f(a)}\in A\times B:a\in A}
            \\&\subseteq A\times B
        \end{align*}
        \\Second, we show $G$ preserves identities.
        \begin{align*}
            G(1_A)&=\set{\pair{a,1_A(a)}\in A\times A:a\in A}
            \\&=\set{\pair{a,a}\in A\times A:a\in A}
        \end{align*}
        \\Third, we show $G$ preserves compostions. Let $f:A\rightarrow B,g:B\rightarrow C$.
        \begin{align*}
            G(g\circ f)&=\set{\pair{a,g\circ f(a)}\in A\times C:a\in A}
            \\&=\set{\pair{a,c}\in A\times C:\exists b(f(a)=b \:\&\: g(b)=c)}
            \\&=\set{\pair{a,c}\in A\times C:\exists b(\pair{a,b}\in G(f) \:\&\: \pair{b,c}\in G(g))}
            \\&=G(f)\circ G(g)
        \end{align*}
    \end{enumerate}
\end{enumerate}
\end{document}